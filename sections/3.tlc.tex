
\section[image=gae_aulenti]{TLC}
\begin{frame}[plain]{}
    \sectionpage
\end{frame}

\begin{frame}{Specification}
    TLC is an explicit-state model checker for \tlap specifications.
    TLC can evaluate specifications that have the standard form
    \[
        Init \land \square \left[ Next \right]_{vars} \land Temporal
    \]

    where
    \begin{itemize}
        \item $Init$ is the initial predicate
        \item $Next$ is the next-state action
        \item $Temporal$ is a temporal formula (usually liveness conditions)
    \end{itemize}

\end{frame}

\begin{frame}{Temporal Formula}
    TLC can evaluate a temporal formula if it is a conjunction of the following classes of formulas:
    \begin{itemize}
        \item State Predicate
        \item Invariant ($\square P$, where $P$ is state a predicate)
        \item Box-Action formula ($\square[A]_v$, where $A$ is an action)
        \item Simple Temporal Formula
    \end{itemize}
\end{frame}

\begin{frame}{Simple Temporal Formula}
    Given
    \begin{itemize}
        \item Simple boolean operators ($\land$, $\lor$, $\neg$, $\implies$, $\equiv$, $TRUE$, $FALSE$)
        \item Temporal state formula, that is a temporal formula obtained from state predicates by applying
        \begin{itemize}
            \item simple boolean operators
            \item temporal operators ($\square$, $\Diamond$, $\leadsto$)
        \end{itemize}
        \item Simple action formula\begin{align*}
            \WF_v \left( A \right) &&
            \SF_v \left( A \right) &&
            \square \Diamond \left< A \right>_v &&
            \Diamond \square \left[ A \right]_v &&
        \end{align*}
    \end{itemize}
    A simple temporal formula is defined as a formula constructed from temporal state formulas and simple action formulas by applying simple boolean operators
\end{frame}

\begin{frame}{TLC modes}
    TLC has two different modes
    \setbeamercovered{transparent}
    \begin{itemize}
        \item<1,3> Model-checking mode
        \only<1> {
            \begin{enumerate}
                \item TLC computes all the initial states by evaluating $Init$
                \item TLC constructs the graph of all states of the system, using a BFS, by evaluating $Next$
                \item While expanding the graph, TLC checks that invariants and constraints are satisfied
                \item TLC checks that the temporal properties are satisfied
            \end{enumerate}
        }
        \item<2-> Simulation mode
        \only<2>{
            \begin{itemize}
                \item TLC repeatedly constructs and checks behaviors of the system that have a fixed maximum length
            \end{itemize}
        }
    \end{itemize}
    \setbeamercovered{invisible}
    \only<3>{
        Note that model-checking is possible only if the graph of the states is finite. If it is not finite, constraints to limit the number of states to be considered by TLC can be specified.
    }
\end{frame}
