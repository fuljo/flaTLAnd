\section[image=bgphoto_cut]{\tlap Introduction}
\begin{frame}[plain]{}
    \sectionpage
\end{frame}

\begin{frame}{Introduction}
    \tlap\ is a high-level specification language for modeling concurrent and distributed digital systems:
    \begin{itemize}
        \item Algorithms
        \item Programs
        \item Complex computing systems
    \end{itemize}

    \tlap is based on set theory, first-order logic and the Temporal Logic of Actions (TLA); it uses ordinary, basic math.
\end{frame}

\begin{frame}{Leslie Lamport}
    \tlap is developed by Leslie Lamport
    \begin{itemize}
        \item Original author of \LaTeX, first release in 1984
        \item Fundamental contribution to the theory of distributed systems
        \begin{itemize}
            \item Logical Clocks
            \item Byzantine General's problem
            \item Chandy-Lamport distributed snapshot algorithm
            \item Paxos algorithm
            \item many, many other contributions
        \end{itemize}
        \item Turing Award in 2013 (and many other prizes)
    \end{itemize}
\end{frame}

\begin{frame}{\tlap ecosystem}
    \setbeamercovered{transparent}
    \begin{description}
        \item<1->[\tlap] specification language
        \item<1->[TLC] model checker and simulator of \tlap specs
        \item<-1>[PlusCal] algorithm language similar to a simple programming language, can be translated to \tlap
        \item<-1>[TLAPS] system for mechanically checking proofs written in TLA
        \item<-1>[TLATeX] pretty-printer to typeset \tlap specifications in \LaTeX
    \end{description}
    \setbeamercovered{invisible}
    \pause
    \vspace{0.5cm}
    \begin{center}
        In this presentation, we will focus on \tlap and TLC
    \end{center}
\end{frame}

%TODO: history?

\begin{frame}{Motivations for simple, high-level language}
    Why should we use an high-level language which uses ordinary and simple math instead of some kind of programming language?
    \setbeamercovered{transparent}
    \begin{itemize}[<+->]
        \item It helps us abstract away from implementation details
        \item No special or new syntax, only simple math
        \item Specification of the system is written before the implementation, design errors are found as early as possible
        \item Specification is independent from the language used for implementation
    \end{itemize}
    \setbeamercovered{invisible}
\end{frame}