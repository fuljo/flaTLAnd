\section[image=clock_col]{First Modelling Example}
\begin{frame}[plain]{}
    \sectionpage
\end{frame}

\begin{frame}{Simple Clock}
    A $24h$ \textbf{Clock} is a device that keeps time, in our example:
    \setbeamercovered{transparent}
    \begin{itemize}
        \item<1-> It goes from \texttt{00:00} to \texttt{23:59}, ignoring seconds.
        \item<2-> After the minute 59 there is minute 0 and the hour is incremented.
        \item<2-> After the hour reached 23, it goes back to 0.
        \item<3-> The clock should start at 00:00.
    \end{itemize}
    \setbeamercovered{invisible}
\end{frame}

\begin{frame}{Modelling Variables}
    We need to define 2 variables:
    \setbeamercovered{transparent}
    \begin{itemize}[<+->]
        \item $hr$ -- The current hour of the clock
        \item $min$ -- The current minute of the clock\demo
    \end{itemize}
    \setbeamercovered{invisible}
    \onslide<+->
    \vspace{1cm}
    The only valid initial state for the clock is expressed with this predicate:
    \[
        hr = 0 \quad \land \quad min = 0
    \]
    \demo
\end{frame}

\begin{frame}{Modelling Actions}
    \begin{block}{Tick Action}
        $Tick \defeq AdvanceMin \,\land AdvanceHr$
    \end{block}
    \pause
    \begin{block}{$AdvanceMin$}
        The next minute ($min'$) is the current one ($min$) plus one. Buf if the minute is already 59, the next is 0.
        \demo
    \end{block}
    \pause
    \begin{block}{$AdvanceHr$}
        \only<3>{
            $hr' = hr + 1$\\
            $\phantom{lol}$\\
            $\phantom{lol}$\\
            $\phantom{lol}$\\
            $\phantom{lol}$\\
        }
        \only<4>{
            $hr' = \IF min < 59$\\
            $\phantom{hr' =}\THEN hr$\\
            $\phantom{hr' =}\ELSE hr + 1$\\
            $\phantom{lol}$\\
            $\phantom{lol}$\\
        }
        \only<5->{
            $hr' = \IF min < 59$\\
            $\phantom{hr' = }\THEN hr$\\
            $\phantom{hr' = }\ELSE \IF hr < 23$\\
            $\qquad\qquad\quad\THEN hr + 1$\\
            $\qquad\qquad\quad\ELSE 0$
            \demo
        }
    \end{block}
\end{frame}

\begin{frame}{Behavior Specification}
    \begin{block}{Option \#1: Init -- Next}
        The model is defined via the predicate for the initial states $Init$, and the only action $Tick$.
        \demo
    \end{block}
    \pause
    \begin{block}{Option \#2: TLA formula}
        Specify the behavior using a TLA formula:

        \[
            Init \land \square \left[Tick\,\right]_{vars} \textcolor{gray!60}{\,\land \ldots}
        \]
        \demo

        Note that $\quad\square \left[Tick\,\right]_{vars} \defeq \square ( Tick \lor vars' = vars\,)$
    \end{block}
\end{frame}

\begin{frame}{Check Temporal Formulae}
    We want to verify some temporal formulae:
    \pause
    \begin{block}{The clock always ticks}
        \[
            AlwaysTick \defeq \square \Diamond \langle Tick\, \rangle _{vars}
        \]
    \end{block}
    \pause
    \begin{block}{All the possible times are reachable}
        \begin{equation*}
            \begin{gathered}
                AllTimes \defeq \forall h \in (0 \ldots 23), m \in (0 \ldots 59):\\
                \square \Diamond (hr = h\; \land\; min = m)
            \end{gathered}
        \end{equation*}
    \end{block}
    \demo
\end{frame}

\begin{frame}{Fixing the Specification}
    Indeed the specified model is not entirely correct:
    \[
        \texttt{00:00} \rightarrow \texttt{00:00} \rightarrow \texttt{00:00} \rightarrow \texttt{00:00} \rightarrow \cdots
    \]
    It allows an infinite sequence of \emph{stuttering steps}, which is not the correct clock behavior.

    \pause
    We need to add a fairness constraint
    \begin{block}{Fairness constraint}
        \[
            LiveClock \defeq Clock \,\land \WF _{vars}(Tick\,)
        \]

        Since $Tick$ is always enabled, it must be \emph{eventually} taken, there must be only a finite number of \emph{stuttering steps} between each $Tick$.
    \end{block}
    \demo
\end{frame}
