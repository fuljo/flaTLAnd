\section[image=bgphoto_cut]{First Modelling Example}
\begin{frame}[plain]{}
    \sectionpage
\end{frame}

\begin{frame}{Simple Clock}
    A \textbf{Clock} is a device that keeps time, in our example:
    \setbeamercovered{transparent}
    \begin{itemize}
        \item<1-> The hours go from 0 to 23
        \item<1-> The minutes go from 0 to 59
        \item<2-> After the minute 59 there is minute 0 and the hour is incremented.
        \item<2-> After the hour reached 23, it goes back to 0.
        \item<3-> The clock should start at 00:00.
    \end{itemize}
    \setbeamercovered{invisible}
\end{frame}

\begin{frame}{Modelling Variables}
    We need to define 2 variables:
    \pause
    \setbeamercovered{transparent}
    \begin{itemize}[<+->]
        \item \texttt{hr} -- The current hour of the clock
        \item \texttt{min} -- The current minute of the clock\demo
    \end{itemize}
    \setbeamercovered{invisible}
    \onslide<+->
    \vspace{1cm}
    The only valid initial state for the clock is expressed with this predicate:
    \[
        \texttt{hr} = 0 \quad \land \quad \texttt{min} = 0
    \]
    \demo
\end{frame}

\begin{frame}{Modelling Actions}
    \begin{block}{Next Action}
        \texttt{Next == NextMin $\land$ NextHr}
    \end{block}
    \pause
    \begin{block}{\texttt{NextMin}}
        The next minute (\texttt{min'}) is the current one (\texttt{min}) plus one. Buf if the minute is already 59, the next is 0.
        \demo
    \end{block}
    \pause
    \begin{block}{\texttt{NextHr}}
        \only<3>{
            \texttt{hr' = hr + 1}\\
            \texttt{\phantom{lol}}\\
            \texttt{\phantom{lol}}\\
            \texttt{\phantom{lol}}\\
            \texttt{\phantom{lol}}\\
        }
        \only<4>{
            \texttt{hr' = IF min < 59}\\
            \texttt{\phantom{hr' = }THEN hr}\\
            \texttt{\phantom{hr' = }ELSE hr + 1}\\
            \texttt{\phantom{lol}}\\
            \texttt{\phantom{lol}}\\
        }
        \only<5->{
            \texttt{hr' = IF min < 59}\\
            \texttt{\phantom{hr' = }THEN hr}\\
            \texttt{\phantom{hr' = }ELSE IF hr < 23}\\
            \texttt{\phantom{hr' = ELSE }THEN hr + 1}\\
            \texttt{\phantom{hr' = ELSE }ELSE 0}
            \demo
        }
    \end{block}
\end{frame}

\begin{frame}{Behavior Specification}
    \begin{block}{Option \#1: Init -- Next}
        The model is defined via the predicate for the initial states (\texttt{Init}), and the only action (\texttt{Next}).
        \demo
    \end{block}
    \pause
    \begin{block}{Option \#2: TLA formula}
        Specify the behavior using a TLA formula:

        \[
            \texttt{Init} \land \square \texttt{[Next]}_{\texttt{vars}} \textcolor{gray!60}{\,\land \ldots}
        \]
        \demo

        Note that $\square \texttt{[Next]}_{\texttt{vars}} \defeq \square ( \texttt{Next} \lor \texttt{vars'} = \texttt{vars} )$
    \end{block}
\end{frame}

\begin{frame}{Check Temporal Formulae}
    We want to verify some temporal formulae:
    \pause
    \begin{block}{The clock always ticks}
        \[
            \texttt{AlwaysTick == } \square \Diamond \langle \texttt{Next} \rangle _{\texttt{vars}}
        \]
    \end{block}
    \pause
    \begin{block}{All the possible times are reachable}
        \begin{equation*}
            \begin{gathered}
                \texttt{AllTimes == } \forall h \in (0 \ldots 23), m \in (0 \ldots 59):\\
                \square \Diamond (\texttt{hr = } h\; \land\; \texttt{min = } m)
            \end{gathered}
        \end{equation*}
    \end{block}
    \demo
\end{frame}

\begin{frame}{Fixing the Specification}
    Indeed the specified model is not entirely correct:
    \[
        \texttt{00:00} \rightarrow \texttt{00:00} \rightarrow \texttt{00:00} \rightarrow \texttt{00:00} \rightarrow \cdots
    \]
    is a valid behavior composed of an infinite sequence of \emph{stuttering steps}.
    \pause
    We need to add a fairness constraint
    \begin{block}{Weak Fairness constraint}
        \[
            \texttt{LiveClock == Clock } \land \texttt{ WF}_{\texttt{vars}}\texttt{(Next)}
        \]

        Since \texttt{Next} is always enabled, it must be \emph{eventually} taken, there must be only a finite number of \emph{stuttering steps} between each \texttt{Next}.
    \end{block}
    \demo
\end{frame}
