\section[image=bgphoto_cut]{Two-phase commit}
\begin{frame}[plain]{}
    \sectionpage
\end{frame}

\begin{frame}
    \frametitle{Specifying two-phase commit}

    Our goal is to model a system with one \alert{transaction manager} and
    several \alert{resource managers} that need to perform a distributed
    transaction.
    The transaction may:
    \begin{itemize}
        \item \textbf{commit} if all resource managers have committed
        \item \textbf{abort} in any other case
    \end{itemize}

    \begin{center} 
        \begin{tikzpicture}[
            >=stealth,
            edge from parent/.style={draw,<->},
            every node/.style={ellipse,draw}
        ]
            \node[rectangle, inner sep=.75em] (tm) {$tm$}
            child foreach \x in {rm1,rm2,rm3} {
                node {$\x$}
            };
        \end{tikzpicture}
    \end{center}
\end{frame}



\subsection{TCommit}
\begin{frame}
    \frametitle{Specifying transaction commit}
    We start with a simple setting, in which we only consider the 
    \alert{resource managers} and their \alert{states}.
    We do \emph{not} model
    \begin{itemize}
        \item the transaction manager
        \item the communication channels
        \item the transaction itself
    \end{itemize}
    \uncover<2->{
    We write the specification in module $TCommit$, which will be refined later

    \begin{center}
        \begin{tlatex}
            \moduleLeftDash{ {\MODULE} $TCommit$}\moduleRightDash
        \end{tlatex}
    \end{center}
    }
\end{frame}

\setbeamercovered{transparent}

\begin{frame}
    \frametitle{Constants and variables}

    We declare $RM$ as the set of all resource managers and $rmState$ as the
    state of each resource manager.
    \begin{tlabox}
        &\CONSTANT RM \\
        &\VARIABLE rmState
    \end{tlabox}

    Every constant value is a set -- even 0, 1 and the string "xyz" --
    but for the former their elements are simply undefined, so we
    can't test if $a \in 0$.

    We will assign elements to $RM$ when performing model checking.
\end{frame}

\begin{frame}
    \frametitle{Type checking}
    \begin{uncoverenv}<1>
        \tlap doesn't provide explicit typing of variables. It is customary to
        define an invariant $TypeOK$ to specify the \alert{domain} of each variable.
    \end{uncoverenv}
    \begin{uncoverenv}<2>
        We specify that $rmState$ is an array indexed by $RM$ with values in
        the set of strings below.
    \end{uncoverenv}
    \begin{uncoverenv}<2->
        \begin{tlabox}
            TCTypeOK \defeq rmState \in
            [RM \rightarrow \{&\str{working}, \str{prepared}, \\
                            &\str{committed},\str{aborted}\}]
        \end{tlabox}
    \end{uncoverenv}
    \begin{uncoverenv}<3>
        You can think of it as a mathematical function
        \[
            rmState: RM \rightarrow \{\str{working}, \dots\}
        \]
        We can refer to a value with $rmState[r]$, where $r \in
        RM$.
    \end{uncoverenv}
\end{frame}

\begin{frame}
    \frametitle{The temporal formula of the specification}
    \begin{uncoverenv}<1->
            We specify the behavior with a temporal formula
        \begin{tlabox}
            TCSpec \defeq TCInit \land \Box [TCNext]_{rmState}
        \end{tlabox}
    \end{uncoverenv}
    \begin{uncoverenv}<1->
        where initially each resource manager is working
        \begin{tlabox}
            TCInit \defeq [ r \in RM \mapsto \str{working}]
        \end{tlabox}
        and at each step either all resource managers remain in their
        current state (stuttering step) or they take a $Next$-step
    \end{uncoverenv}
\end{frame}

\begin{frame}
    \frametitle{Evolution of the system}

    \begin{columns}[c]
    \begin{column}{0.65\textwidth}
        \uncover<1>{What is a possible step of our system?}

        \uncover<2,3>{A resource manager may}
        \begin{itemize}
            \item<2> Finish working and become prepared
            \item<3> Decide to commit or abort
        \end{itemize}
    \end{column}
    \begin{column}{0.35\textwidth}
        \centering
        \scalebox{.75}{
            \begin{tikzpicture}[
                >=stealth,
                sibling distance=7em,
                edge from parent/.style={draw,->},
                every node/.style={ellipse,draw}
            ]
                \node (w) {$working$}
                child {
                    node (p) {$prepared$}
                    child {
                        node[inner xsep=0] (c) {$committed$}
                    }
                    child {
                        node (a) {$aborted$}
                    }
                };
                \path [draw, ->] (w.south east) to [bend left=30] (a.north);
            \end{tikzpicture}
        }
    \end{column}
    \end{columns}
    \vspace{\baselineskip}
    \begin{tlabox}
        TCNext \defeq \uncover<2,3>{\E r \in RM :}
        \uncover<2>{Prepare(r)} \uncover<3>{\lor Decide(r)}
    \end{tlabox}
\end{frame}

\begin{frame}
    \frametitle{The Prepare action}

    Resource manager $r$ evolves from \str{working} to \str{prepared},
    while all the other ones remain in the same state.
    \begin{tlabox}
        Prep&are(r) \defeq \\
            &\uncover<1>{%
            \land rmState[r] = \str{working}} \\
            &\uncover<2>{%
            \land rmState' = [rmState \EXCEPT \bang[r] = \str{prepared}]}
    \end{tlabox}

\end{frame}

\begin{frame}
    \frametitle{The Decide action}

    \only<1-3>{%
    First subformula: transition \alert{$prepared \rightarrow committed$}}
    \only<4-6>{%
    Second subformula: transition \alert{$working, prepared \rightarrow aborted$}}
    \begin{tlabox}
        De&cide(r) \defeq \\
            &\lor \uncover<1>{%
                \land rmState[r] = \str{prepared}} \\
            &\phantom{\.{\lor}} \uncover<2>{%
                \land canCommit} \\
            &\phantom{\.{\lor}}\uncover<3>{%
                \land rmState' = [rmState \EXCEPT \bang[r] = \str{committed}]} \\
            &\lor \uncover<4>{%
                \land rmState[r] \in \{\str{working}, \str{prepared}\}} \\
            &\phantom{\.{\lor}}\uncover<5>{
                \land notCommitted} \\
            &\phantom{\.{\lor}}\uncover<6>{%
                \land rmState' = [rmState \EXCEPT \bang[r] = \str{aborted}]}
    \end{tlabox}

    \begin{onlyenv}<1-3>
        \begin{tlabox}
            \uncover<2>{%
            canCommit \defeq \A r \in RM : rmState[r] \in \{\str{prepared},
            \str{comm.}\}}
        \end{tlabox}
    \end{onlyenv}

    \begin{onlyenv}<4-6>
        \begin{tlabox}
            \uncover<5>{%
            notCommitted \defeq \A r \in RM : rmState[r] \neq \str{committed}}
        \end{tlabox}
    \end{onlyenv}

\end{frame}

\begin{frame}
    \frametitle{Checking the specification}

    Once we have modeled how the system \emph{behaves}, we can use TLC to check
    if desired properties hold.

    In particular we are interested in checking that \alert{no two RMs have
    arrived at conflicting decisions}. This \emph{safety property} can be
    expressed with the invariant $TCConsistent$:

    \begin{tlabox}
        TC&Consistent \defeq \\
        &\A r1, r2 \in RM : \neg
            \land rmState[r1] = \str{aborted} \\
        &\phantom{\A r1, r2 \in RM : \.{\neg}}
            \land rmState[r2] = \str{committed}
    \end{tlabox}

\end{frame}

\setbeamercovered{invisible}

\begin{frame}
    \frametitle{Title}

    \begin{center} 
        \begin{tikzpicture}[
            >=stealth,
            sibling distance=7em,
            edge from parent/.style={draw,->},
            every node/.style={ellipse,draw}
        ]
            \node (w) {$working$}
            child {
                node (p) {$prepared$}
                child {
                    node[inner xsep=0] (c) {$committed$}
                }
                child {
                    node (a) {$aborted$}
                }
            };
            \path [draw, ->] (w.south east) to [bend left=30] (a.north);
        \end{tikzpicture}
    \end{center}

\end{frame}
