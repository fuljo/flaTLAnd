\section[image=bgphoto_cut]{Two-phase commit}
\begin{frame}[plain]{}
    \sectionpage
\end{frame}

\begin{frame}
    \frametitle{Specifying two-phase commit}

    Our goal is to model a system with one \alert{transaction manager} and
    several \alert{resource managers} that need to perform a distributed
    transaction.
    The transaction may:
    \begin{itemize}
        \item \textbf{commit} if all resource managers have committed
        \item \textbf{abort} in any other case
    \end{itemize}

    \begin{center} 
        \begin{tikzpicture}[
            >=stealth,
            edge from parent/.style={draw,<->},
            every node/.style={ellipse,draw}
        ]
            \node[rectangle, inner sep=.75em] (tm) {$tm$}
            child foreach \x in {rm1,rm2,rm3} {
                node {$\x$}
            };
        \end{tikzpicture}
    \end{center}
\end{frame}



\subsection{TCommit}
\begin{frame}
    \frametitle{Specifying transaction commit}
    We start with a simple setting, in which we only consider the 
    \alert{resource managers} and their \alert{states}.
    We do \emph{not} model
    \begin{itemize}
        \item the transaction manager
        \item the communication channels
        \item the transaction itself
    \end{itemize}
    \uncover<2->{
    The specification will be written in module $TCommit$, which will be
    refined later

    \begin{center}
        \begin{tlatex}
            \moduleLeftDash{ {\MODULE} $TCommit$}\moduleRightDash
        \end{tlatex}
    \end{center}
    }
\end{frame}

\setbeamercovered{transparent}

\begin{frame}
    \frametitle{Constants and variables}

    We declare $RM$ as the set of all resource managers and $rmState$ as the
    state of each resource manager.
    \begin{tlabox}
        &\CONSTANT RM \\
        &\VARIABLE rmState
    \end{tlabox}

    Every constant value is a set -- even 0, 1 and the string "xyz" --
    but for the former their elements are simply undefined, so we
    can't test if $a \in 0$.

    We will assign elements to $RM$ when performing model checking.
\end{frame}

\begin{frame}
    \frametitle{Type checking}
    \tlap doesn't provide explicit typing of variables. It is customary to
    define an invariant $TypeOK$ to specify the \alert{domain} of each variable.
    
    \uncover<2->{
    \begin{tlabox}
        TCTypeOK \defeq rmState \in
        [RM \rightarrow \{&\str{working}, \str{prepared}, \\
                        &\str{committed},\str{aborted}\}]
    \end{tlabox}
    }

\end{frame}

\setbeamercovered{invisible}

\begin{frame}
    \frametitle{Title}

    \begin{center} 
        \begin{tikzpicture}[
            >=stealth,
            sibling distance=7em,
            edge from parent/.style={draw,->},
            every node/.style={ellipse,draw}
        ]
            \node (w) {$working$}
            child {
                node (p) {$prepared$}
                child {
                    node[inner xsep=0] (c) {$committed$}
                }
                child {
                    node (a) {$aborted$}
                }
            };
            \path [draw, ->] (w.south east) to [bend left=30] (a.north);
        \end{tikzpicture}
    \end{center}

\end{frame}
